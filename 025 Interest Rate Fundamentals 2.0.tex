%\documentclass[aspectratio=169]{beamer} %1920x1080 resolution
\documentclass[handout, aspectratio=169]{beamer}
%\documentclass[handout]{beamer} % For printing only 
\usepackage{../../../LatexTemplatesAndSamples/mastering_ir_derivatives_beamer} % My custom style

% This goes together with handout feature above
%\pgfpagesuselayout{2 on 1}

% Needs to be set for each class
\subtitle{Interest Rate Fundamentals}

\begin{document}

\input{../../../LatexTemplatesAndSamples/firstThreeSlides.tex}

\section{Term Structure of Interest Rates and Bonds}
\begin{frame}{Derive Spot Rates based on Bond Prices - 1}
	\pause
	\begin{itemize}
		\item Bond prices are expressions of interest rates 
		\item Zero-coupon bonds pay only par at maturity
	\end{itemize}
	\begin{example}
		\scriptsize
		\begin{itemize}
			\item Assume we have six zero-coupon bonds with different maturities
			\item Each bond has a face value of \$100 and matures at the end of each year for the next six years
			\item The transaction price for each of these bonds is listed below
		\end{itemize}
		\begin{columns}
			\begin{column}{0.5\textwidth}
				\begin{table}[h]
					\begin{tabular}{|c|c|} 
						\hline
						\textbf{Maturity} & \textbf{Price}\\				
						\hline
						1 year & 94.88 \\
						2 years & 90.75 \\
						3 years & 87.25 \\
						\hline
					\end{tabular}
				\end{table}
			\end{column}
			\begin{column}{0.5\textwidth}
				\begin{table}[h]
					\begin{tabular}{|c|c|} 
						\hline
						\textbf{Maturity} & \textbf{Price}\\				
						\hline
						4 years & 84.05 \\
						5 years & 80.78 \\
						6 years & 77.63 \\
						\hline
					\end{tabular}
				\end{table}
			\end{column}
		 \end{columns}
	\end{example}
\end{frame}

\begin{frame}{Derive Spot Rates based on Bond Prices - 2}
	\begin{example}
		\scriptsize
		continued
		\begin{itemize}
			\item Define term and compounding convention for spot rate
			\item We assume an annual rate and annual compounding
			\item Using standard discounting we can determine the spot rates, $\{r_i\}$
			\item $P = \frac{\$100}{(1 + r)^n}$ 
			\item where $P$ is the bond price, $n$ the number of years
		\end{itemize} 
		\begin{columns}
			\begin{column}{0.5\textwidth}
				\begin{table}[h]
					\begin{tabular}{|c|c|c|} 
						\hline
						\textbf{Maturity} & \textbf{Price} & \textbf{Rate (\%)}\\				
						\hline
						1 year & 94.88 & 5.40 \\
						2 years & 90.75 & 4.97 \\
						3 years & 87.25 & 4.65 \\
						\hline
					\end{tabular}
				\end{table}
			\end{column}
			\begin{column}{0.5\textwidth}
				\begin{table}[h]
					\begin{tabular}{|c|c|c|} 
						\hline
						\textbf{Maturity} & \textbf{Price} & \textbf{Rate (\%)}\\				
						\hline
						4 years & 84.05 & 4.44 \\
						5 years & 80.78 & 4.36 \\
						6 years & 77.63 & 4.31 \\
						\hline
					\end{tabular}
				\end{table}
			\end{column}
		\end{columns}
	\end{example}
\end{frame}

\section{Compounding}
\begin{frame}{Compounding}
	\begin{itemize}
		\item Compounding is the process where the interest earned is reinvested to generate additional earnings
		\item Compounding usually happens multiple times and create exponential growth in interest earned
		\item Various compounding conventions exist
		\begin{itemize}
			\item Annual, semi-annual, quarterly, monthly, weekly
			\item Daily is now the most common convention due to LIBOR cessation
			\item LIBOR has been discontinued and replaced by SOFR
			\item SOFR is an overnight rate, hence daily compounding is natural
			\item Continuous compounding is also common in quantitative models
		\end{itemize}
		\item Conversions from one rate to another can be explored as an exercise
	\end{itemize}
\end{frame}

\begin{frame}{Compounding Intuition - 1}
	\pause
	\begin{itemize}
		\item Let's create some intuition around compounding
		\item Assume an annual rate and quarterly compounding
		\item $P_i$ is the principal after $i$ compounding periods
		\item $r$ is the annual rate, $I$ is the interest
		\item Let's illustrate how much principal we have after one year  
		\begin{example}
			\scriptsize
			After 3 months, which is the end of the first compounding period, we have:
			\begin{align*}
				\onslide<7->{P_1 &= P_0 + I \\}
				\onslide<8->{&= P_0 + P_0 \cdot \frac{r}{4} \\}
				\onslide<9->{&= P_0 \cdot \left(1+\frac{r}{4}\right)}
			\end{align*}
		\end{example}
	\end{itemize}
\end{frame}

\begin{frame}{Compounding Intuition - 2}
	\begin{example}
		\tiny
		After 6 months, which is the end of the second compounding period, we calculate the value of the principal using the compound interest formula:
		\begin{itemize}
			\item[$P_0$:] Initial principal
			\item[$P_1$:] Value at end of first compounding period
			\item[$P_2$:] Value at end of second compounding period
		\end{itemize}
		\begin{align*}
			\onslide<5->{P_2 &= P_1 + I \\}
			\onslide<6->{	&= P_1 + P_1 \cdot \frac{r}{4} \\}
			\onslide<7->{	&= P_1 \cdot \left(1+\frac{r}{4}\right) \\}
			\onslide<8->{	&= P_0 \cdot \left(1+\frac{r}{4}\right) \cdot \left(1+\frac{r}{4}\right) \\}
			\onslide<9->{	&= P_0 \cdot \left(1+\frac{r}{4}\right)^2}
		\end{align*}
	\end{example}
\end{frame}

\begin{frame}{Compounding Intuition - 3}
	\begin{example}
		\tiny
		After 9 months, which corresponds to the end of the third compounding period, the final amount $P_3$ is calculated as follows:
		\begin{itemize}
			\item[$P_0$:] Initial principal
			\item[$P_2$:] Value at end of second compounding period
			\item[$P_3$:] Value at end of third compounding period
		\end{itemize}
		\begin{align*}
			\onslide<5->{P_3 &= P_2 + I \\}
			\onslide<6->{	&= P_2 + P_2 \cdot \frac{r}{4} \\}
			\onslide<7->{	&= P_2 \cdot \left(1+\frac{r}{4}\right) \\}
			\onslide<8->{	&= P_0 \cdot \left(1+\frac{r}{4}\right)^2 \cdot \left(1+\frac{r}{4}\right) \\}
			\onslide<9->{	&= P_0 \cdot \left(1+\frac{r}{4}\right)^3}
		\end{align*}
	\end{example}
\end{frame}

\begin{frame}{Compounding Intuition - 4}
	\begin{example}
		\tiny
		Finally, after 1 year we get
		\begin{itemize}
			\item[$P_0$:] Initial principal
			\item[$P_3$:] Value at end of third compounding period
			\item[$P_4$:] Value at end of fourth and last compounding period
		\end{itemize}
		\begin{align*}
			\onslide<5->{P_4 &= P_3 + I \\}
			\onslide<5->{	&= P_3 + P_3 \cdot \frac{r}{4} \\}
			\onslide<5->{	&= P_3 \cdot (1+\frac{r}{4}) \\}
			\onslide<5->{	&= P_0 \cdot (1+\frac{r}{4})^3 \cdot (1+\frac{r}{4}) \\}
			\onslide<5->{	&= P_0 \cdot (1+\frac{r}{4})^4}
		\end{align*}
	\end{example}
\end{frame}

\begin{frame}{Compounding Intuition - 5}
	\begin{corollary}
		We can now generalize and state that for $n$ compounding periods
		\begin{align*}
			P_n &= P_0 \cdot \left(1+\frac{r}{n}\right)^n
		\end{align*}
		\begin{itemize}
			\item[$r$:] Annual rate
			\item[$n$:] Number of coumpounding periods
			\item[$P_0$:] Initial principal
			\item[$P_n$:] Value at end of n\textsuperscript{th} compounding period
		\end{itemize}
	\end{corollary}
\end{frame}

\section{Basis and Cashflow Calculation}
\begin{frame}{Basis Example}
	\pause
	\begin{itemize}
		\item We start with an example to illustrate the significance of the basis
	\end{itemize}
	\begin{Example}
		\scriptsize
		\onslide<3->{Imagine we have a financial instrument with an interest rate period that starts on 5th January 2023 and ends on 5th July 2023}
		
		\onslide<4->{For a Money Market trade with a 1\% coupon, we calculate the cashflow as follows:
		\begin{align*}
			Cashflow 	&= \$1,000,000 \cdot 1\% \cdot \frac{181}{360} = \$5,027.78
		\end{align*}
		}
		\onslide<5->{For a Treasury bond trade with a 1\% coupon, we calculate the cashflow as follows:
		\begin{align*}
			Cashflow 	&= \$1,000,000 \cdot 1\% \cdot \frac{180}{360} = \$5,000
		\end{align*}
		}
		\onslide<6->{The basis differs, defining how to calculate the length of the interest period}
	\end{Example}
\end{frame}

\begin{frame}{Basis and Cashflow Calculation}
	\pause
	\begin{itemize}
		\item Day count or basis defines the year fraction between two dates
		\item Measure is necessary to define calculation of cash flows
	\end{itemize}
	\begin{block}{}
		\scriptsize
		We calculate cash flows for an interest rate period as
		\begin{align*}
			\text{Cashflow} &= \text{Notional} \cdot \text{Rate} \cdot \text{Year Fraction}\\
			\text{Year Fraction} &= \frac{\text{Basis Numerator}}{\text{Basis Denominator}}
		\end{align*}
		where numerator and denominator are defined by the basis
	\end{block}
\end{frame}

\begin{frame}{Numerator, Denominator and Sample Basis}
	\pause
	\begin{itemize}
		\footnotesize
		\item Measure for basis in numerator calculates days between the start and end date
		\begin{itemize}
			\footnotesize
			\item Actual number of days
			\item Assumed 30 days per month
			\item Business days only
		\end{itemize}
		\item Measure for basis in denominator calculates days in a year
		\begin{itemize}
			\footnotesize
			\item 360 days
			\item 365 days
			\item 365 days with leap year adjustments
			\item 252 days (business days)
		\end{itemize}
		\item Many bases are used in the market, including
		\begin{itemize}
			\footnotesize
			\item Act/360 - used in SOFR swap and money market calculations
			\item 30/360 - used in swap fixed leg and corporate bond calculations
			\item Act/Act - used in treasury bond calculations
			\item BUS/252 - common in Brazil 
		\end{itemize}
	\end{itemize}
\end{frame}

\section{Spot, Zero and Short Rates}
\begin{frame}{Spot and Zero Rates}
	\pause
	\begin{itemize}
		\item A \textbf{spot rate} is the interest rate from today for a set period
		\item \textbf{Zero rates} (denoted as $z$) are continuously compounded spot rates with a basis of A/365 
		\item Hence, the discount factor using zero rates is
	\end{itemize}
	\begin{block}{}
		\begin{align*}
			DF(T_0,T_1) &= e^{-z \cdot \frac{ActualDays(T_0,T_1)}{365}} = e^{-z \cdot dt}
		\end{align*}
		\begin{itemize}
			\item[$DF$:] Discount factor
			\item[$z$:] Zero rate 
		\end{itemize}
	\end{block}
\end{frame}

\begin{frame}{Short Rates}
	\pause
	\begin{itemize}
		\item An \textbf{instantanous short rate} is a spot rate for an infinitesimally short period of time
		\item Provides a foundational building block in continuous-time stochastic interest rate models for quants and traders
		\item Requires understanding of stochastic calculus
		\item The Heath-Jarrow-Morton (HJM) model uses stochastic instantaneous short rate 
		\item Instantaneous short rates are usually continuously compounded
		\item Such rates are not observed but are used as theoretical constructs
		\item Continous nature allows derivation of complex models
	\end{itemize}
\end{frame}

\section{Forward Rates}
\begin{frame}{Forward rates - 1}
	\pause
	\begin{itemize}
		\item Forward rates are set today but apply to specified future periods
		\item Essential for swaps, alongside discount factors
 	\end{itemize}
	\begin{Example}
		\begin{itemize}
			\item Let's assume today's date is 3-Jan-2023
			\item Today's 3-month rate 6 months forward applies from 3-Jul-2023 to 3-Oct-2023
			\item We define it as $f(\text{03-Jan-2023}, \text{03-Jul-2023}, \text{3-Oct-2023})$
			\item More generally, we can denote it as $f(T_0, T_1, T_2)$
		\end{itemize}
	\end{Example}
\end{frame}

\begin{frame}{Forward rates - 2}
	\begin{itemize} 
		\item Forward rates span specific start and end dates
		\item Determined in markets
		\item Market product for these is the Forward Rate Agreement (FRA)
		\item Consider them as the slope or first derivative of spot rates
	\end{itemize}
\end{frame}

\input{../../../LatexTemplatesAndSamples/last_slides_summary_and_questions.tex}

% \begin{frame}{Enhancements to this Lecture}
% 	\begin{itemize}
% 		\item Add Rolling Conventions such as F, MF, P\\
% 		This comes up later in curves where we have a 1M MM maturity date from goes from Jan 31 as it's rolled back
% 		\item Add main calendars such as NY, NYL, BMA
% 		\item Modify principal concept $P_i$ to a Notional concept which is more applicable to derivatives
% 	\end{itemize}
% \end{frame}

\end{document}
